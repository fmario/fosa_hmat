\chapter{Lineare Algebra}

\section{Vektorraum}

Ein Vektorraum $V$ besteht aus einer Menge von Vektoren und den zwei Operationen:
\begin{enumerate}
	\item Vektoren können addiert/subtrahiert werden
	\item Vektoren können mit einem Skalar multipliziert werden
\end{enumerate}
Die Ergebnisse aus den beiden Operationen müssen wieder im Vektorraum liegen.
\\\\
Weitere Bedingungen:
\begin{itemize}
	\item Es gibt einen Nullvektor $0\in V$ mit $v+0=v$ für alle Vektoren $v\in V$
	\item Zu jedem Vektor $v \in V$ gibt es einen inversen Vektor $-v\in V$ mit $v-v=0$
\end{itemize}

\section{Lineare Unabhängigkeit, Basis und Dimension}
\subsection{Linearkombination von Vektoren:}
\[
	v = \lambda_1 \cdot v_1 + \lambda_2 \cdot v_2 + \ldots + \lambda_n \cdot v_n
\]
~\\
\begin{footnotesize}
	$v_1,v_2,\ldots,v_n \in V$\\
	$\lambda_1,\lambda_2,\ldots,\lambda_n \in \mathds{R}$\\
\end{footnotesize}

\subsection{Lineare Unabhängigkeit}
$n$ Vektoren sind linear unabhängig, wenn
\[
	\lambda_1 \cdot v_1 + \lambda_2 \cdot v_2 + \ldots + \lambda_n \cdot v_n = 0
\]
nur mit
\[
	\lambda_1 = \lambda_2 = \ldots = \lambda_n = 0
\]
erreicht ist.\\
\begin{center}
\boxed{\mbox{$n$ Vektoren in $\mathds{R}^m$ sind linear abhängig, falls $n>m$.}}
\end{center}

\subsection{Basis, Dimension und Koordinaten}
Die \textit{Basis} eines Vektorraums ist die Menge vn Vektoren $v_1,v_2,\ldots,v_n$
(\textit{Basisvektoren}), die:
\begin{enumerate}
	\item linear unabhängig sind
	\item den gesamten Vektorraum durch Linearkombination erzeugen
\end{enumerate}

\begin{center}
\boxed{\mbox{Die Dimension des Vektorraums $\mathds{R}^n$ ($\dim V$) ist $n$}}
\end{center}
Die \textit{Koordinaten} sind die Koeffizienten $\lambda_1$ bis $\lambda_n$ der Linearkombination
der Basen, um den gewünschten Vektor zu erreichen.
\\\\
Polynom vom Grad $n$ hat die Basis:
\[ \mathds{P}:\lbrace x^n,x^{n-1}\ldots,x,1 \rbrace \qquad\Rightarrow \dim\mathds{P}_n=n+1 \]

\subsection{Untervektorraum}
\textbf{Definition:} Ein Unterraum ist eine Teilmenge eines Vektorraums, die selbst
einen Vektorraum bildet.\\
\\
\textbf{Satz:} Ist $V$ ein Vektorraum mit Dimension $n$, so gibt es für jedes
$k\in\lbrace 0,1,\ldots,n\rbrace$ einen Unterraum der Dimension $k$.

\section{Lineare Abbildung zwischen Vektorräumen}
Eine Abbildung
\[
	f:V_1 \longrightarrow V_2
\]
zwischen Vektorräumen $V_1$ und $V_2$ ist \textit{linear}, falls
\begin{enumerate}
	\item $f(\lambda\cdot v) = \lambda\cdot f(v)$ für jeden Vektor $v\in V_1$ und jeden
		Skalar $\lambda\in\mathds{R}$
	\item $f(v+w)=f(v)+f(w)$ für alle Vektoren $v,w \in V_1$
\end{enumerate}
~\\
Ist $f: V_1 \longrightarrow V_2$ eine lineare Abbildung, so bleibt der Nullpunkt fest:
\[ f(0) = 0 \]
\textbf{Achtung:} Umkehrung gilt nicht. D.h. $f(0)=0$ bedeutet nicht, dass $f$ linear ist.

\subsection{Matrix einer linearen Abbildung}
Für die Abbildung $f:\mathds{R}^n \rightarrow \mathds{R}^m$ mit den Basisvektoren
$e_1,e_2,\ldots,e_n$ gibt es eine ($m\times n$)-Matrix \textbf{A}, so dass
\[ f(x) = \textbf{A} \cdot x \]
Die Spalten der Matrix \textbf{A} sind die Bilder $v_i = f(e_i)$ der Einheitsvektoren.
\\
Die Abbildung ist umkehrbar, wenn
\[ \det(\textbf{A}) \neq 0 \]
\\\\
\textbf{Satz}\\
Jeder linearen Abbildung $f:\mathds{R}^n \rightarrow\mathds{R}^m$ kann auf eindeutige Weise
eine $m\times n$-Matrix \textbf{A} zugeordnet werden, wobei deren Spalten der Reihe nach die
Bilder $f(e_1),f(e_2),\ldots,f(e_n)$ der Standardbasis des $\mathds{R}^n$ sind.\\
\\
Ist umgekehrt eine $m\times n$-Matrix mit Spalten $v_1,v_2,\ldots,v_n\in\mathds{R}^m$,
so ist die zugehörige lineare Abbildung durch die Vorschrift gegeben:
\[ f: \begin{pmatrix} 	x_1 \\ x_2 \\ \vdots \\ x_n \end{pmatrix} \longmapsto 
	\sum_{i=1}^{n}x_i\cdot v_i \]
	
\subsection{Kern und Bild}
Der \textit{Kern} der Abbildung $f$ besteht aus der Menge der Vektoren in $V_1$ die durch
$f$ auf den Nullvektor in $V_2$ abgebildet werden:
\[ \ker(f) := \lbrace x\in V_1|f(x)=0 \rbrace \]
~\\
Das \textit{Bild} der Abbildung $f$ ist die Menge der möglichen Bildvektoren in $V_2$:
\[ \text{im}(f) := \lbrace y \in V_2|\text{es gibt ein }x\in V_1 \text{ mit } f(x)=y \rbrace \]
~\\
\textit{Kern} und \textit{Bild} von $f$ sind Untervektorräume von $V_1$ resp. $V_2$.
\\\\
\textbf{Dimensionssatz:} Das Bild und der Kern einer linearen Abbildung $f:V_1 \rightarrow V_2$
stehen über den Dimensionssatz in Beziehung:
\[ \dim(V_1) = \dim(\ker(f)) + \dim(\text{im}(f)) \]
~\\
\textbf{Umkehrbarkeit von linearen Abbildungen:} Falls die Dimensionen der betrachteten
Räume $V_1$ und $V_2$ einer linearen Abbildung $f:V_1 \rightarrow V_2$ identisch sind,
können die folgenden äquivalenten Aussagen zur Umkehrbarkeit von $f$ gemacht werden:
\begin{enumerate}
	\item[i)] Die Abbildung $f$ ist umkehrbar
	\item[ii)] $\dim(\ker(f))=0$
	\item[iii)] $\dim(\text{im}(f)) = \dim(V_2)$
\end{enumerate}

\subsection{Lineare Gleichungssysteme als lineare Abbildungen}
Das Gleichungssystem mit $n$ Gleichungen und $n$ Unbekannten $x_1,x_2,\ldots,x_n$ der Form
\[\begin{aligned}
	a_{11}x_1 + a_{12}x_2 + \ldots + a_{1n}x_n &= b_1 \\
	a_{21}x_1 + a_{22}x_2 + \ldots + a_{2n}x_n &= b_2 \\
	\vdots &= \vdots\\
	a_{n1}x_1 + a_{n2}x_2 + \ldots + a_{nn}x_n &= b_n
\end{aligned}\]
kann in Matrizenform als
\[ \begin{pmatrix}
		a_{11} & a_{12} & \ldots & a_{1n} \\
		a_{21} & a_{22} & \ldots & a_{2n} \\
		\vdots &				&				 & \vdots \\
		a_{n1} & a_{n2} & \ldots & a_{nn} \\
	\end{pmatrix} \cdot \begin{pmatrix}
		x_1 \\ x_2 \\ \vdots \\ x_n
	\end{pmatrix} = \begin{pmatrix}
		b_1 \\ b_2 \\ \vdots \\ b_n
	\end{pmatrix}\]
oder kurz
\[ \textbf{A} \cdot x = b \]
geschrieben werden.\\
\\
\textbf{Lösbarkeit linearer Gleichungssystemen\\}
Fall 1: ist $\det\textbf{A}\neq 0$ so ist $x$ eindeutig durch $\textbf{A}^{-1}\cdot b$
gegeben.\\
Fall 2: ist $\det\textbf{A}= 0$ so hat man entweder
\begin{itemize}
	\item[i)] keine Lösung, falls $b \notin\text{bild}\textbf{A}$
	\item[ii)] unendlich viele Lösungen, falls $b \in\text{bild}\textbf{A}$
\end{itemize}

\section{Eigenwertprobleme}
\subsection{Definition}
Gegeben sein ein quadratische Matrix \textbf{A}. finde einen Vektor $x\neq 0$ und einen
Skalar $\lambda$, so dass
\[ \textbf{A} \cdot x = \lambda \cdot x \]

\subsection{Berechnung der Eigenvektoren und Eigenwerte}
\[ (\textbf{A}-\lambda\cdot\textbf{E})\cdot x=0 \]
Dabei muss gelten:
\[ \det(\textbf{A}-\lambda\cdot\textbf{E}) = 0 \]
Nach der Bestimmung der Eigenwerte $\lambda_1,\lambda_2,\ldots,\lambda_n$ kann für
jedes $\lambda_i$ ein Vektor $x_i\neq 0$ mit
\[ (\textbf{A}-\lambda_i\cdot\textbf{E}) \cdot x_i = 0 \]
gefunden werden.

\subsection{Sätze zu Eigenwerten / -vektoren}
\textbf{A} sei eine $n\times n$-Matrix.\\
\begin{itemize}
\item[\textbf{Satz 1:}] \textbf{A} kann höchstens $n$ Eigenwerte haben.\\
\item[\textbf{Satz 2:}] Die Determinante von \textbf{A} ist gleich dem Produkt all ihrer
	Eigenwerte. \[ \det A =  \lambda_1 \cdot \lambda_2 \cdot \ldots \cdot \lambda_n \]
Ist ein Eigenwert = 0, so ist die Matrix singulär (nicht invertierbar).
\item[\textbf{Satz 3:}] $\lim\limits_{n\rightarrow\infty}\textbf{A}=0$ falls für alle
	Eigenwerte $\lambda_i$ gilt $|\lambda_i|<1$
\item[\textbf{Satz 4:}] Bei einer Diagonalmatrix oder Dreiecksmatrix sind die
	Diagonaleinträge gerade die Eigenwerte der Matrix.
\end{itemize}