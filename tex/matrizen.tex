\chapter{Matrizen Repetition}
\section{Grundbegriffe}
$m\times n$-Matrix:
\[ \textbf{A} = (a_{ij}) = \begin{pmatrix}
	a_{11} & a_{12} & \ldots & a_{1n} \\
	a_{21} & a_{22} & \ldots & a_{2n} \\
	\vdots & \vdots & \ddots & \vdots \\
	a_{m1} & a_{m2} & \ldots & a_{mn} \\
\end{pmatrix} \]
~\\
\begin{footnotesize}
\begin{tabular}{ll}
	$a_{ij}:$	&	Matrixelement der Zeile $i$ und der Spalte $j$\\
	$m:$			& Anzahl Zeilen \\
	$n:$			& Anzahl Spalten
\end{tabular}
\end{footnotesize}
~\\
Transponierte Matrix $\textbf{A}^T$ von $A$:
\[ (a_{ij})^T = a_{ji} \]
Für das Produkt der Transponierten gilt:
\[ (\textbf{A}\cdot\textbf{B})^T=\textbf{B}^T\cdot\textbf{A}^T \]

\subsection{Spezielle Matrizen}
\begin{itemize}
	\item \textbf{Nullmatrix:} Alle Elemente der Matrix sind Null, $a_{ij}=0\ \forall i,j$
	\item \textbf{Spaltenmatrix:} Die Matrix enthält nur eine Spalte
		\[\textbf{A}=\begin{pmatrix} a_{11} \\ a_{21} \\ \vdots \\ a_{n1} \end{pmatrix}\]
	\item \textbf{Zeilenmatrix:} Die Matrix enthält nur eine Zeile
		\[\textbf{A}=\begin{pmatrix} a_{11} & a_{12} & \ldots & a_{1m} \end{pmatrix}\]
	\item \textbf{Quadratische Matrix:} Gleiche Anzahl an Zeilen und Spalten.
	\item \textbf{Symmetrische Matrix:} Quadratische Matrix, welche mit ihrer Transponierten
		übereinstimmen ($\textbf{A}=\textbf{A}^T$).
		\[ a_{ij} = a_{ji}\ \forall i,j \]
	\item \textbf{Dreiecksmatrix:} Alle Elemente die unter oder über der Hauptdiagonalen stehen
		sind Null.
	\item \textbf{Diagonalmatrix:} Alle Elemente ausserhalb der Hauptdiagonalen sind Null.\\
		Spezialfall ist die \textbf{Einheitsmatrix E}
\end{itemize}

\section{Grundoperationen}
Addition und Subtraktion zweier Matrizen \textbf{A} und \textbf{B}:
\[ \textbf{C}=\textbf{A}\pm\textbf{B}\qquad\Leftrightarrow\qquad c_{ij}=a_{ij}\pm b_{ij} \]
~\\
Multiplikation mit einem Skalar:
\[ \textbf{B}=\lambda\cdot\textbf{A}\qquad\Leftrightarrow\qquad b_{ij}=\lambda\cdot a_{ij} \]
~\\
Matrizenmultiplikation einer $m\times n$-Matrix \textbf{A} und einer $n\times p$-Matrix \textbf{B}:
\[ c_{ij} = \sum_{k=1}^{n}a_{ik}\cdot b_{kj} \qquad \textbf{C}_{m\times p} \]

\section{Die Inverse Matrix}
Definition:
\[ \textbf{A}\cdot\textbf{A}^{-1} = \textbf{E} \]
~\\
Die Inverse Matrix existiert (ist \textit{invertierbar, regulär}), wenn:
\[ \det\textbf{A}\neq 0 \]
