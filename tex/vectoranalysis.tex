\chapter{Vektroanalysis}
\section{Felder}
\subsection{Skalarfelder}
Eine skalare Grösse $\Phi$ die jedem Raumpunkt $\vec{r}=\vec{r}(x,y,z)$
zugeordnet ist, heisst \textit{Skalarfeld:}
\[ \Phi = \Phi(\vec{r}) = \Phi(x,y,z) \]
Wenn die Werte der Funktion $\Phi$ nur vom Abstand $r$ von einem Zentrum abhängen,
heisst $\Phi(r)$ ein \textit{zentrales} oder \textit{radiales} Feld.\\
\\
Linien in 2D resp. Flächen in 3D auf denen das Skalarfeld konstant ist werden
\textit{Äquipotentiallinien} resp. -flächen genannt (bzw. \textit{Niveaulinien/-fälchen}).

\subsection{Vektorfelder}
Einem Vektorfeld $\vec{F}$ ist jedem Raumpunkt $\vec{r}=(x,y,z)$ einen Vektor
$\vec{F}(x,y,z)$ zugeordnet.
\[ \vec{F} = \begin{pmatrix} F_x(x,y) \\ F_y(x,y) \end{pmatrix} \]
~\\
\paragraph{Gravitationsfeld}
Die Kraft auf eine Masse $m$ an der Stelle $\vec{r}=(x,y,z)$ ist gegeben durch:
\[ \vec{F}_m = -G\frac{M\cdot m}{r^2} \vec{e}_r \]
~\\
\begin{footnotesize}
\begin{tabular}{ll}
	$r=\sqrt{x^2+y^2+z^2}$:		& Abstand der beiden Masse\\
	$\vec{e}_r=\frac{1}{\sqrt{x^2+y^2+z^2}}\begin{pmatrix}x\\y\\z\end{pmatrix}$:
	& radialer Einheitsvektor von $M$ in Richtung $m$	\\
	$G=6.67428\cdot 10^{-11}\,{\frac{m^3}{kg\cdot s^2}}$:	& universelle Gravitationskonstante\\
	$M=5.97\cdot 10^{24}\,$kg:	&	Erdmasse
\end{tabular}
\end{footnotesize}

\subsection{Arbeit und Wegintegrale}
Gegeben sei ein Vektorfeld $\vec{F}$ und ein Weg $C$ paramtetrisiert durch
$t\in[t_1,t_2] \mapsto \vec{\gamma}(t)$ mit Anfangspunkt $\vec{\gamma}(t_1)$ und
Endpunkt $\vec{\gamma}(t_2)$, dann ist das \textit{skalare Kurvenintegral, Wegintegral}
oder die \textit{Arbeit} des Vektorfeldes $\vec{F}$ längs der Kurve $C$ gegeben durch:
\[ W = \int_C \vec{F}\cdot\di\vec{r}=\int_{t_1}^{t_2}\vec{F}(\vec{\gamma}(t)) \cdot 
	\dot{\vec{\gamma}}(t)\di t \]
Ist das Vozeichen von $W$ positiv, so wird Arbeit oder Energie gewonnen, ist es negativ,
muss Arbeit oder Energie reingesteckt werden.

\subsection{Parametrisierung einer Kurve}
Ist die Kurve durch eine Funktionsvorschrift $y=f(x)$ gegeben, kann
\[ \vec{y}(t) = \begin{pmatrix} t \\ f(t) \end{pmatrix} \]
für die Kurve gewählt werden.
\paragraph{Kreis mit Radius $r > 0$ im Ursprung:}
\[ \vec{y}(t) = r\begin{pmatrix} \cos(t) \\ \sin(t) \end{pmatrix} \]
wobei $t\in[0,2\pi]$
\paragraph{Archimedische Spirale:}
\[ \vec{y}(t) = \frac{a}{2\pi}t\begin{pmatrix} \cos(t) \\ \sin(t) \end{pmatrix} \]
mit $t\in[0,k\cdot 2\pi]$, wobie $k$ der Anzahl Windungen und $a$ dem Abstand der einzelnen
Windungen entspricht.
\paragraph{Lemniskate ($\infty$):}
\[ \vec{y}(t) = a\begin{pmatrix} \cos(t)\sqrt{2\cos(2t)} \\ \sin(t)\sqrt{2\cos(2t)} \end{pmatrix} \]


\subsection{Gradient enes Skalarfeldes}
Der Gradient eines Skalarfeldes $\Phi$ ist definiert als:
\[ \grad\Phi(x,y,z)=\vec{\nabla}\Phi = \begin{pmatrix}
	\pdifrac{\Phi}{x} \\ \pdifrac{\Phi}{y} \\ \pdifrac{\Phi}{z} \end{pmatrix} \]
	
\subsection{Konservative Felder}
Fall es für ein Vektorfeld $\vec{F}$ ein Skalarfeld $\Phi$ mit $\nabla\Phi = \vec{F}$
gibt, so ist das Linienintegral längs des Weges $C$ von $\vec{r}_1$ bis $\vec{r}_2$ nur
vom Anfangs- und Endpunkt abhängig und ist gegeben durch:
\[ \int_C \vec{F}\cdot\di\vec{r} = \int_C \vec{\nabla}\Phi\cdot\di\vec{r}
	= \Phi(\vec{r}_2) - \Phi(\vec{r}_1) \]
	
\paragraph{Pfadunabhängigkeit:}
Sind $C_1$ und $C_2$ verschiedene Pfade mit gleichem Start- und Endpunkt so ist:
\[ \int_{C_1} \vec{\nabla}\Phi\cdot\di\vec{r} = \int_{C_2} \vec{\nabla}\Phi\cdot\di\vec{r} \]	

\paragraph{Geschlossene Kurven:} Ist $C$ eine geschlossene Kurve dann gilt:
\[ \oint_C \vec{\nabla}\Phi\cdot\di\vec{r} =0 \]

\subsection{Kriterium für konservative Felder}
Das Vektorfeld
\[ \vec{F} = \begin{pmatrix} F_x(x,y) \\ F_y(x,y) \end{pmatrix} \]
ist konservativ, wenn
\[ \pdifrac{F_x}{y} = \pdifrac{F_y}{x} \]
erfüllt ist. Das Potential $\Phi$ ergibt sich durch
\[\begin{aligned}
		\Phi(x,y) &= \int F_x(x,y)\di x +f_1(y)\\
		\Phi(x,y) &= \int F_y(x,y)\di y +f_2(x)\\
\end{aligned}\]
Die unbekannten Funktionen $f_1(y)$ und $f_2(x)$ erhält man durch Vergleichen der beiden
Integrale.

\paragraph{Prüfen mit $\rot$}~\\
Ein Vektorfeld $\vec{F}$ ist konservativ, falls die Rotation verschwindet
\[ \rot\vec{F} = \vec{0} \]
\[ \rot\vec{F}(x,y,z) = \begin{pmatrix}
	\pdifrac{F_z}{y} - \pdifrac{F_y}{z}\\
	\pdifrac{F_x}{z} - \pdifrac{F_z}{x}\\
	\pdifrac{F_y}{x} - \pdifrac{F_x}{y}\\
	\end{pmatrix} \]
Für den 2D Fall kann die $z$-Komponente auf Null gesetzt werden.
